\section{Exercise 1: Auto-vectorization}

\begin{frame}{}
\begin{center}
{\Huge Exercise 1: Auto-vectorization}
\end{center}
\end{frame}

\begin{frame}{Which kinds of loops can be vectorized automatically?}
Following conditions must be fulfilled:
\begin{itemize}
\item Only innermost loops
\item No prohibiting data dependencies
\item Unit stride: Loop increment by one only
\item Supported operations and data types in the loop body
\item Countable: The number of iterations must be known and fixed when entering the loop at runtime. No data-dependent exits. \enquote{Single entry and a single exit}.
\item Straight-line code: The control flow of every iteration must be the same. Exception: Masked assignments (selective writing).
\item No function calls. Exceptions: Intrinsic math functions and \enquote{compatible} functions.
\end{itemize}
\end{frame}

\begin{frame}{Which datatypes and operations are allowed in order to enable auto-vectorization of loops?}
\begin{center}
\begin{Large}
SSE Specific:
\end{Large}
\end{center}

Integers:
\begin{itemize}
	\item 8-, 16- and 32-bit: Most arithmetic and logical operators
	\item Exception: No direct support for division!
	\item Many mathematical functions: \texttt{sqrt}, \texttt{min}, \texttt{max}, \texttt{abs}
	\item 64-bit: limited support
\end{itemize}
32- and 64-bit floats: 
\begin{itemize}
	\item +, -, *, /, min, max and sqrt
	\item Functions from a vectorized math library, e.g. trigonometric functions.
\end{itemize}
\end{frame}

\begin{frame}{Which types of dependency analysis do the compiler perform?}
\begin{enumerate}
\item Proven dependencies/Loop-carried dependencies:
\begin{itemize}
\item Read-after-write (flow dependency): Possibly reading too early. Cannot safely be vectorized.
\item Write-after-read (anti dependency): Can mostly be safely vectorized (if there is no other use of that memory location in the loop body).
\item Write-after-write (output dependency): This can generally not be safely vectorized.
\end{itemize}
Exception: Some reduction idioms (e.g. multiply-accumulate) recognized by the compiler are safely vectorized.
\item Potential dependencies/Pointer aliasing:
The compiler may check at compile or runtime (by inserting some checker code) for overlapping memory regions caused by pointer aliasing.
\end{enumerate}
\end{frame}

\begin{frame}{How does programming style influence auto-vectorization?}
Things to avoid:
\begin{itemize}
	\item Global pointers
	\item Manual loop optimizations
	\item Indirect addressing
	\item Misaligned data
	\item Arrays of structures (AoS)
\end{itemize}
\end{frame}

\begin{frame}{Is there a way to assist the compiler through language extensions? If yes, please give details.}
\begin{itemize}
\item \texttt{\#pragma novector}
\item \texttt{\#pragma ivdep}
\item \texttt{\#pragma loop count (n)}
\item \texttt{\#pragma vector always}
\item \texttt{\#pragma vector aligned}
\item \texttt{\#pragma vector nontemporal}
\item restrict
\end{itemize}
\end{frame}

\begin{frame}{Which loop optimizations are performed by the compiler in order to vectorize and pipeline loops?}
\begin{itemize}
\item Loop sectioning (Strip-mining)
\item Loop blocking
\item Loop interchanging
\item Loop peeling
\item Loop collapsing
\end{itemize}
\end{frame}
