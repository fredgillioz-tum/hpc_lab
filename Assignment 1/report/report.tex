\documentclass[11pt]{article}

\usepackage[margin=.8in,top=1.1in,bottom=1.1in]{geometry} % page layout
\usepackage{amsmath,amsthm,amssymb,amsfonts} % math things
\usepackage{graphicx} % include graphics
\usepackage{fancyhdr} % header customization
\usepackage{titlesec} % help with section naming
\usepackage{csquotes}

\usepackage[T1]{fontenc}
\usepackage[utf8]{inputenc}

% headers
\pagestyle{fancy} 
\fancyhf{} % clear all
\fancyhead[L]{\sffamily\small Lab Course: High Performance Computing --- Assignment Report}
\fancyhead[R]{\sffamily\small Page \thepage}
\renewcommand{\headrulewidth}{0.2pt}
\renewcommand{\footrulewidth}{0.2pt}
\markright{\hrulefill\quad}

\newcommand{\hwhead}[4]{
\begin{center}
\sffamily\large\bfseries Lab Course: High Performance Computing Assignment Report #1
\vspace{2mm} 
\normalfont

#2 -- #3 -- \texttt{#4}
\end{center}
\vspace{6mm} \hrule \vspace{4mm}
}

% ------------------------------------------------------------------------------
% Start here -- Fill in your name, imat and email
% ------------------------------------------------------------------------------

\newcommand{\name}{Frédéric Gillioz} %
\newcommand{\imat}{03657002} %
\newcommand{\email}{frederic.gillioz@tum.de} %

\begin{document}

% ------------------------------------------------------------------------------
% Change xx (and only xx) to the current sheet number
% ------------------------------------------------------------------------------
\hwhead{1}{\name}{\imat}{\email}

% ------------------------------------------------------------------------------
% Fill in your solutions
% ------------------------------------------------------------------------------

\setcounter{section}{-1}
\section{General}

Problems:
\begin{itemize}
\item Intel compiler 2017 installation on Ubuntu: Extracted installer package may not be located in a path containing a space character otherwise the installation procedure fails at the licensing step.
\end{itemize}

\section{Auto-vectorization}

\subsubsection*{Which kinds of loops can be vectorized automatically?}
Innermost loops.

\subsubsection*{Which datatypes and operations are allowed in order to enable auto-vectorization of loops?}

\subsubsection*{Which types of dependency analysis do the compiler perform?}

pointer aliasing
dependencies between loop iterations

\subsubsection*{How does programming style influence auto-vectorization?}
Programming style can inhibit vectorization. For example the existence of global pointers prevents the compiler to prove that there is no aliasing to a memory location which is subject to auto-vectorization. For the same reason one should make use of array notations instead of pointers.



Within the innermost loop, it is advisable to use straight-line code (no jumping, i.e. no \texttt{switch}, \texttt{goto} or \texttt{return} statements), use arrays as L-values only (constant expressions as R-values are ok) and to have only assignments. Simple \texttt{if} branches are possible if they can be treated as \textit{masked assignments}, i.e. an instruction is applied to the whole SIMD register but the resulting scalars are written back selectively.

Discouraged are calls to any functions (except math library calls), use of non-vectorizable operations and mixing of vectorizable data types.

Also the loop exit condition is not allowed to depend on the own loop's data. It must be invariant within the loop.

Dependencies between loop iterations are not allowed.

Manual loop unrolling and splitting a loop with a large body into multiple smaller loops should be avoided.

Access memory efficiently:
• Favor inner loops with unit stride.
• Minimize indirect addressing.
• Align your data to 16-byte boundaries (for Intel ® SSE instructions).

Use structure of arrays (SoA) instead of array of structures (AoS).

obstacles:
Non-contiguous memory access: non-unit stride or indirect addressing


\subsubsection*{Is there a way to assist the compiler through language extensions? If yes, please give details.}
In some cases, certain keywords or directives may be applied in the code for auto-vectorization to occur.

restrict (aber eigentlich keine lang extension)

\subsubsection*{Which loop optimizations are performed by the compiler in order to vectorize and pipeline loops?}
Loop unrolling.
Loop peeling?

\end{document}
